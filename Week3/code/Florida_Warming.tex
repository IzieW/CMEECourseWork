\documentclass[11pt]{article}
\usepackage{graphicx}
\usepackage[a4paper, total={6in, 8in}]{geometry}
\usepackage{sectsty}
\sectionfont{\large}
\graphicspath{{../results/}}

\title{\vspace{-3cm}\Large Increasing temperatures in the Florida Keys over the last century}
\author{\vspace{-1cm}\normalsize Izie Wood}
\date{\normalsize}
\begin{document}
\maketitle{\large}
\vspace{-1cm}
\section{Introduction}
        The  increasing concentration of green house gases
    in our atmosphere is understood to produce a warming effect. This paper studies a rise in temperature measured in the Florida keys 
    over the last century, and finds a positive correlation between temperature increase
    over time. Florida is getting warmer. 
\section{Method}
        Temperatures were recorded in Key West, Florida USA annually over the last century. 
    To see the relationship between temperature over time, I conducted a permutation analysis, calculating the 
    pearson's correlation coefficient between observed temperatures and years, and comparing 
    it to a completely random distribution of correlation coefficients, obtained through
    randomely shuffling the data over 5000 trials. 
\section{Results}
            Results revealed a moderate positive correlation between temperature and time
        (r(98) = 0.53). Out of 5000 randomely generated correlation coefficients, none of them were 
        greater than our observed correlation coefficient. Thus, the observed positive correlation between temperature and time has an 
        approximate asymptotic p-value of 0.00.   
        \begin{figure}[ht]
        \begin{center}
            \includegraphics[scale=0.30]{Permutation_results}
        \end{center}
        \end{figure}
\section{Conclusion}
    Analysis conducted here suggests that Florida temperatures are increasing, likely
    a result of the changing atmospeheric concentrations of green house gases. Florida is 
    getting warmer.

\end{document}